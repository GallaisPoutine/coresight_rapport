%--------------------------------------------
% PARTIE 1 : Problématique et enjeux du sujet
%--------------------------------------------

\section{Problématique et enjeux du sujet}
\label{chp:part1:issue_concern}

Grace à l'évolution de nouvelles technologies, on assiste à une
miniaturisation exponentielle des composants et une explosion de leurs
capacités. Cette croissance se retrouve dans le monde des semiconducteurs où
de nombreuses solutions se développent chaque année, répondant à divers
besoins techniques, plus particulièrement, les microprocesseurs et
microcontrôleurs. \\

Ce sont des systèmes sur puces suffisamment miniaturisés pour associer
plusieurs composants au sein d'un seul boitier. Ces Systems on Chip (SoCs)
exécutent des jeux d'instructions à la manière d'un processeur d'ordinateur.
On les retrouve dans les objets du quotidiens, tels que dans l'électroménager
ou la téléphonie mobile : des systèmes dont les capacités étaient jusqu'alors
réservées aux ordinateurs. Ces microprocesseurs présentent donc une grande
diversité dans leurs domaines d'applications. Leur présentation universelle,
mais surtout des technologies à un faible coût de revient, expliquent par
ailleurs leur production quasiment exponentielle d’année en année et leur
présence récurrente dans notre quotidien. Entre autres, on recense une
production de plusieurs millions d'unités par mois. \\

% https://thewalkingdeadfrance.org/global-microprocessor-market-covid-updates-3/
% https://www.thesneaklife.com/2020/02/06/apercu-du-marche-microprocesseur-mondial-2019-2026-intel-qualcomm-apple-amd-freescale/
% https://news.knowledia.com/CH/fr/articles/rapport-sur-le-marche-unite-de-microprocesseur-serveur-mpu-mondial-2020-37a1c04d7002644204e0163c3b8842d3e608c9c9?source=rss
% https://www.globenewswire.com/news-release/2020/02/07/1981844/0/en/Global-Microprocessors-Market-Insights-2015-2030-Green-Evolution-Has-Become-an-Urgent-Priority-for-Wireless-NSPs-and-Microprocessor-Manufacturing-Companies.html

% µProc market share: https://www.semiconductors.org/annual-semiconductor-sales-increase-21.6-percent-top-400-billion-for-first-time/
% https://www.globenewswire.com/news-release/2020/02/07/1981844/0/en/Global-Microprocessors-Market-Insights-2015-2030-Green-Evolution-Has-Become-an-Urgent-Priority-for-Wireless-NSPs-and-Microprocessor-Manufacturing-Companies.html

Le SoC STM32MP1 est la première famille de microprocesseurs commercialisée par
la société STMicroelectronics. Afin de démontrer ses capacités sur le marché,
ce microprocesseur doit faire valoir toutes ses caractéristiques. Cette
mission que l'on m'a confiée répond à un besoin interne chez
STMicroelectronics.  En effet, dans la version actuelle du microprocesseur
sont présentés des blocs hardwares non exploitables. Ces composants matériels
sont dormants en raison des pilotes (ou drivers) dont deux processeurs de la
carte ne supportaient pas jusqu'à présent.  Pourtant ces IPs sont
technologiquement les seuls moyens permettant de déboguer dans l'espace noyau.
Jusqu'à présent des solutions non conventionnelles pour investiguer des points
sur le SoC étaient présentées. Il s'agit donc d'améliorer les capacités de
débogage de l'environnement en déployant les outils communautaires et de
proposer aux clients un moyen plus ergonomique de déboguer. \\

Comme le SoC développé par STMicroeletronics possède la capacité de
fonctionner avec un OS de type Linux, il est donc soumis aux règles de la
communauté Linux. L'implémentation d'une nouvelle fonctionnalité au sein de la
carte éléctronique est également un moyen de contribution sur la communauté
Open Source. Cela permettrait donc d'enrichir le SoC et d'affirmer
STMicroelectronics comme acteur du noyau Linux. \\

À moyen et long termes, cette mission pourrait économiquement servir à gagner
des clients tant en marché de masse qu'à des entreprises consommatrices de
semiconducteurs ciblées, puisque cela servirait à déboguer plus facilement et
avec efficience. En effet, les clients vont naturellement se tourner vers des
solutions ergonomiques où il sera plus simple de développer et déboguer une
application. Ainsi s'ils ne peuvent pas le faire sur le SoC STM32MP15, ils
risquent de se tourner vers d'autres solutions, et à terme, l'entreprise
risque de perdre des clients. \\
