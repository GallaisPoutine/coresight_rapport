%-----------------------------
% PARTIE 3 : Rapport personnel
%-----------------------------

\section{Point sur le coronavirus}
\label{sec:coronavirus}

Vis-à-vis de la situation extraordinaire et imprévue devant laquelle le monde
s'est trouvé sans ressources apparentes, je pense qu'un point est nécessaire
afin d'expliquer comment le site de ST Le Mans a géré cette crise, la manière
dont la société a réagi pour protéger ses employés et la manière dont elle a
impacté le déroulement de ce stage. Cela m'a également permis de voir comment
une société met en place une gestion de crise plus précisément. En effet, j'ai
eu un apport très théorique et superficiel de la gestion de crise lors de ma
formation à l'ESEO. Un bénéfice que cela a eu sur moi aura été de mieux
comprendre comment s'articule une entreprise dans de telles circonstances. \\

Au moment du premier discours d'Emmanuel Macron annonçant le confinement de la
population, le 16 Mars au soir, une cellule de crise avait déjà été mise en
place chez ST Le Mans. Elle était composée de M. Bourgeais et M. Dubois,
responsable sécurité, ainsi que de la DRH Mme. Huguet. Par chance, le discours
étant tombé un lundi soir, cela a permit au chef d'équipe M. Peurichard de
faire un essai de communiqué par téléphone et Teams en nous précisant les
démarches à suivre pour partir en confinement. \\

La crise sanitaire et économique a impliqué que je télétravaille.
L'entreprise a autorisé aux employés d'emporter les moyens nécessaires pour
travailler, donc j'ai pris l'ordinateur portable sur lequel je développe ainsi
qu'une STM32MP15 et leurs câbles d'alimentation respectifs. Un VPN interne à
l'entreprise a été mis en place pour assurer un accès aux documents internes
et des visioconférences les lundis après-midi une semaine sur deux également.
L'utilisation d'un Teams a permis de contacter différentes personnes ou
équipes de l'entreprise. \\

En interne, STMicroelectronics a mis en place un certain nombre de règles et
de mesures. Par exemple, une caméra thermique a été installée à l'entrée du
site, la température de chaque employé est vérifiée par les hôtesses
d'accueil. En cas de température trop élevée, ou plus généralement de
suspicion de contamination, la personne en question est priée de contacter la
cellule de crise, puis de prendre un rendez-vous chez un médecin afin de se
faire tester. L'agencement des bureaux a été modifié pour garantir au moins
1,5m entre chaque employé, et les bureaux face-à-face se sont vu mettre en
place une plaque de plexiglas pour limiter les aérosols. Enfin, pour permettre
un distanciation sociale, comme le gouvernement le préconisait, les accès aux
portes furent améliorés, de sorte que l'on ait simplement à les pousser ou
tirer en limitant tous contacts. Des masques et des lotions hydroalcooliques
furent distribués le premier jour de retour sur le site. \\

Cette situation de télétravail a agi comme catalyseur d'autonomie et de prise
de décision. Malgré une communication régulière avec Christophe mon tuteur et
les collègues, la résolution de problèmes passait d'abord par une
introspection; et c'est en ultime recours que je m'adressais à mon tuteur via
la messagerie d'entreprise. La crise est notamment arrivée pendant l'état de
l'art du débogage interprocesseur. Donc, une certaine créativité a dû être
mise en place pour orienter les recherches et étaler le spectre des
possibilités.

%Ce fut une situation un peu anxiogène selon moi, car entre 6 et 8 membres de
%l'entreprises ont contracté la maladie, dont le directeur du site Jérôme
%Bourgeais ainsi qu'un collêgue stagiaire que j'ai cotoyé régulièrement avant
%et après la situation de crise. Ce n'est qu'au début du confinement que la
%cellule de crise a averti les principales personnes en contact avec ce
%stagiaire qu'il était contaminé. \\

%Les locaux sont organisés par équipe et généralement en openspace. Un espace
%pour les deux stagiaires de l'équipe (dont moi) a été aménagé. Nous avons à
%notre disposition une carte de développement ainsi qu'un ordinateur portable
%et un écran auxiliaire. Je suis convié aux weekly de l'équipe ainsi qu'aux
%meetings généraux. Je contacte réguliérement les membres de l'équipe pour
%m'aider à la résolution d'un problème. \\
