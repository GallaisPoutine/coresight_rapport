%-----------------------------
% PARTIE 3 : Rapport personnel
%-----------------------------

\section{Apport personnel}
\label{sec:personal_benefits}

Ce stage m'a amené dans les retranchements de mes connaissances que ce soit
sur le plan de la gestion ou de la technique. En effet, ce mode de
développement m'était jusqu'alors totalement inconnu. Le monde de l'open
source autorise des discussions avec des experts hors de l'entreprise de
travail. Ainsi les interactions sont internationales et non fermées : l'open
source est un monde d'échanges et de discussions. \\

En terme de connaissances techniques, la durée de cette expérience
professionnelle aura épaissie mon portfolio de connaissances déjà acquises, et
m'aura fait entrevoir de nouvelle technologies. L'immersion dans le kernel
Linux m'a permis de mieux concevoir comment Linux fonctionne en interne et
plus généralement comment un système complexe s'organise. C'est par ailleurs
un point que j'ai beaucoup apprécié car je me suis beaucoup intéressé à Linux.
Ce stage m'a fait approfondir le maniement d'outils auxiliaires tels que GIT
ou bien grep pour rechercher rapidement à travers un code source conséquent.
Travailler au sein d'un projet quasiment exclusivement codé en C m'a également
permit d'entrevoir les subtilités de son langage. Cette exposition au sein du
kernel Linux m'a finalement fait arborer des modèles de conception d'un driver
Linux et de la description du composant qu'il gère. Enfin pour terminer ce
paragraphe sur une note d'humour, et au risque de me faire haïr par mon tuteur
Christophe, j'ai toujours utilisé les éditeurs de texte Nano et Gedit. Ces
logiciels n'étant pas implémentés nativement sur la STM32MP15, j'ai été
contraint de prendre en main le logiciel d'édition Vim, et délaisser (sans
regrets) les précédents éditeurs. \\

Enfin, dans le cadre du consortium des logiciels libres pour les processeurs
ARM avec Linaro, un séminaire "Advanced Kernel Debug Training" de 4 sessions
en visioconférence durant le mois de mai a été proposé aux employés. Cela a
été extrèmement enrichissant. Premièrement avec le lien direct concernant mon
sujet de stage. Deuxièmement car j'ai pu avoir un retour d'expérience de
Daniel Thompson, un professionnel de chez Linaro, où j'ai pu appliqué
instantanement ses conseils. \\

En plus des connaissances théoriques et pratiques liées à l'apprentissage
pendant le stage, cette expérience professionelle m'a permis de découvrir le
monde de l'Open Source et comment une entreprise s'organise autour de ces
projets. Par ailleurs, ayant une vision bien définie de la recherche à cause
de mon entourage et de mon stage précédent en recherche appliquée à
Polytechnique j'ai eu une vision neuve sur la R\&D. J'ai réalisé qu'en
entreprise, la recherche s'accompagne beaucoup de développement pour arriver à
un produit ou une solution valide.

