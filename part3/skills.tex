%-----------------------------
% PARTIE 3 : Rapport personnel
%-----------------------------

\section{Savoir-être requis et utilisés}

En tant que partie intégrante d'une équipe, et plus généralement dans une
entreprise l'ingénieur doit faire preuve d'un ensemble de qualités
personnelles. \\

Le premier savoir-être primordial au sein d'une équipe est la communication.
Par exemple, lors de l'explication d'un problème pendant les réunions ou par
mail avec Mathieu Poirier dont les échanges étaient réguliers, il m'a fallu
faire preuve d'un esprit de synthèse pour fournir une certaine abstraction à
un problème technique bien précis. \\

ST étant une entreprise internationale, toutes les traces écrites telles que
les comptes-rendu, exposés ou documentations sont rédigés en anglais. Un
vocabulaire technique est nécessaire pour comprendre et rédiger la
documentation publique. \\

De même, au niveau organisationnel, j'ai dû prendre un certain nombre de
décisions, notamment lors de l'étude de faisabilité pour établir les choix
possibles et décider s'il était judicieux de continuer à avancer à l'encontre
du temps disponible et restant. L'initiative est aussi une clé du métier
d'ingénieur. C'est pourquoi au cours du développement de mon stage, j'ai été
amené à faire des propositions sur l'orientation de la mission du stage.
Notamment lorsque j'ai découvert des erreurs d'intégration entre perf et
le framework Coresight. \\

Enfin, j'ai dû faire preuve d'une adaptation rigoureuse. Premièrement
vis-à-vis de la situation de crise et le changement soudain d'environnement de
travail, et deuxièmement avec les normes de développement du noyau Linux que
je ne connaissais pas et auxquelels j'ai dû m'accomoder. Les différents
résultats obtenus m'ont demandé une permanente remise en doute par rapport aux
résultats et un questionnement constant des différentes méthodes utilisées. \\
