%---------------------------
% Fiche de synthèse du stage
%---------------------------

\section*{Fiche de synthèse du stage}
\addcontentsline{toc}{section}{Fiche de synthèse du stage}
\label{pref:internship_synthesis}

\begin{table}[H]
\resizebox{\textwidth}{!}{%
\begin{tabular}{|c|c|}
\hline
Intitulé du stage &
  Coresight / Plateforme de Débogage \\ \hline
Date de début &
  02/03/2020 \\ \hline
Date de fin &
  28/08/2020 \\ \hline
Entreprise d'accueil &
  STMicroelectronics \\ \hline
Principales personnes impliquées &
  \begin{tabular}[c]{@{}c@{}}Christophe Roullier : Maître de stage\\ Mathieu Poirier : Mainteneur Coresight\\ Gérald Baéza et Loïc Pallardy : Architectes STM31MP1\\ Raphaël Gallais-Pou : stagiaire\end{tabular} \\ \hline
\begin{tabular}[c]{@{}c@{}}Planning résumé\\ (Réalisations personnelles)\end{tabular} &
  \begin{tabular}[c]{@{}c@{}}Prise en main de l'environnement\\ Mise en place du décodage de traces ARM Coresight\\ État de l'art du débogage interprocesseurs\\ Implémentation d'une fonctionnalité Coresight (TPIU puis ETMv3)\\ Validation de la fonctionnalité \\ Rédaction d'une documentation sur le framework\end{tabular} \\ \hline
Contraintes et difficultés rencontrées &
  \begin{tabular}[c]{@{}c@{}}État de l'art\\ Comparatif des versions des composants Coresight\\ Prise en main du framework perf\\ Implémentation de la fonctionnalité\end{tabular} \\ \hline
Résultats &
  \begin{tabular}[c]{@{}c@{}}Étude de faisabilité implémentable mais non concluante\\ Intégration de la fonctionnalité au kernel ST\\ Page Wiki disponible en interne\end{tabular} \\ \hline
\end{tabular}%
}
\end{table}
