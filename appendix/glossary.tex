%----------
% Glossaire
%----------

\chapter*{Glossaire}
\addcontentsline{toc}{chapter}{Annexe 2 : Glossaire}

\newcommand{\bu}[1]{\textbf{\underline{#1}}}

\noindent
\bu{API : } (Application Programming Interface) Bibliothèque de fonctions et
attributs fournis pour des programmes. \\

\noindent
\bu{ARM : } Société développant des processeurs et microcontrôleurs
d'architecture 32 et 64 bits. \\

\noindent
\bu{Bit : } Unité ne pouvant prendre que deux valeurs : 0 ou 1. \\

\noindent
\bu{Bogue : } Défaut d'exécution ou de conception d'un programme informatique.
\\

\noindent
\bu{Boot : } voir Démarrage. \\

\noindent
\bu{Bootchain : } Séquence d'évènements permettant au noyau Linux de
s'initialiser. \\

\noindent
\bu{Bus : } Système communicatif d'échanges de données entre plusieurs
composants. \\

\noindent
\bu{Composant : } Élément électronique de base assemblé pour créer un ensemble
de fonctions électroniques et informatiques plus complexes. \\

\noindent
\bu{Context ID : } Paquet Coresight permettant de déterminer le contexte
d'exécution d'une instruction. \\

\noindent
\bu{Coresight : } Jeu de composants fournis par ARM pour déboguer. \\

\noindent
\bu{CPU : } (Central Processoring Unit) Processeur, composant exécutant les
instructions machines des programmes informatiques. \\

\noindent
\bu{DDR : } (contraction de DDR SDRAM, Double Data Rate Synchronous Dynamic
Random Access Memory) Type de mémoire RAM. \\

\noindent
\bu{Dégogage : } Action d'utiliser un débogueur. \\

\noindent
\bu{Débogueur : } Logiciels permettant d'inspecter point par point les
instructions d'un programme et d'en révéler les bogues. \\

\noindent
\bu{Décodeur : } Logiciel reconstituant des signaux chiffrés ou compressés. \\

\noindent
\bu{Démarrage : } Ensemble de processus visant à rendre le noyau Linux
fonctionnel. \\

\noindent
\bu{Device tree : } Description matérielle d'une puce électronique et de ses
périphériques. \\

\noindent
\bu{Driver : } Programme informatique bas niveau visant à intéragir avec un
périphérique ou un composant. \\

\noindent
\bu{ETM : } (Embedded Trace Macrocell) Composant source générant des traces
Coresight. \\

\noindent
\bu{FSBL : } (First Stage Boot Loader) Logiciel intervenant en premier dans la
séquence de démarrage. C'est celui qui charge les périphériques principaux et
le SSBL. \\

\noindent
\bu{Firmware : } Microcode intégré dans un matériel électronique pour qu'il
puisse fonctionner. \\

\noindent
\bu{Framework : } Infrastructure logicielle conçue pour faciliter le
développement d'un logiciel. \\

\noindent
\bu{HAL : } (Hardware Abstraction Layer) Interface entre la couche matérielle
(les composants) d'un système et la couche logicielle. \\

\noindent
\bu{Horodatage : } Association d'un évènement à une date, relative ou absolue.
\\

\noindent
\bu{Image : } Exécutable statique contenant le noyau Linux sous forme d'objet.
\\

\noindent
\bu{Interface : } Dispositif mis en place pour autoriser la communication
entre deux éléments d'un système, limite commune à deux systèmes. \\

\noindent
\bu{IP : } (Internal Peripheral ou Périphérique Interne) Composant, élément d'un
ensemble électronique. \\

\noindent
\bu{JTAG : } (Joint Test Action Group) Standard de ports électroniques et de
communication destiné à la simplification du test de cartes électroniques. \\

\noindent
\bu{Kernel : } voir noyau. \\

\noindent
\bu{Masque : } Donnée utilisée pour modifier un groupe de bits. \\

\noindent
\bu{Memory map : } Projection en mémoire de composants ou périphériques,
décrivant l'adresse et la taille du composant auquel elle est associée. \\

\noindent
\bu{minicom : } Logiciel informatique émulant un terminal dans le but de
communiquer par liaison série. \\

\noindent
\bu{MMU : } (Memory Management Unit) Composant d'un processeur permettant la
virtualisation des adresses mémoires. \\

\noindent
\bu{MPU : } (Micro Processing Unit) Processeur dont les éléments ont été
réduits pour rentrer dans un seul boitier. \\

\noindent
\bu{Noeud : } Dans le device tree, description matérielle d'un composant,
ensemble de propriétés visant à définir un composant. \\

\noindent
\bu{Noyau : } Partie du système d'exploitation gérant les processus et
ressources de l'ordinateur. \\

\noindent
\bu{Open Source : } Logiciels dont les critères correspondent à ceux définis
pas l'Open Source Initiative, et dont le code source est accessible au
public. Cela ne signifie pas que le code source est modifiable ou gratuit
systématiquement. \\

\noindent
\bu{OpenCSD : } (Open CoreSight Decoder) Décodeur Open Source de traces
Coresight. \\

\noindent
\bu{OS : } (Operating System) voir Système d'exploitation. \\

\noindent
\bu{Paquet : } Ensemble d'informations numériques formant un message. \\

\noindent
\bu{perf : } Logiciel informatique de supervision d'évènements, basé sur le
composant PMU. \\

\noindent
\bu{PMU : } (Performance Monitoring Unit) Composant intégré dans les
processeurs, capturant des évènements se déroulant dans un processeur. \\

\noindent
\bu{Propriété : } Dans le device tree, champ décrivant une caractéristique du
composant qu'il intègre. \\

\noindent
\bu{RAM : } (Random Access Memory) Mémoire vive informatique, dont le contenu
est effacé à chaque démarrage. \\

\noindent
\bu{Registre : } Case mémoire permettant de stocker de petites quantités
d'informations. Ces registres peuvent être au sein d'un processeur ou dans un
composant. \\

\noindent
\bu{ROM : } (Read Only Memory) Mémoire informatique non volatile, dont le
conenu est fixé. \\

\noindent
\bu{Semiconducteur : } Matériau à base de silicium ayant des propriétés
intermédiaires entre l'isolant et le conducteur élèctrique. \\

\noindent
\bu{SoC : } (System on Chip) Semiconducteur, système embarqué miniaturisé au
sein d'une seul boitier électronique. \\

\noindent
\bu{SSBL : } (Second Stage Boot Loader) Logiciel intervenant dans la séquence
de démarrage, permettant de charger le noyau Linux. \\

\noindent
\bu{Système d'exploitation : } Ensemble de programmes gérants les ressources
de l'ordinateur et l'interface entre l'utilisateur et le noyau. \\

\noindent
\bu{TF-A : } (Trusted Firmware - ARM) Logiciel fourni par ARM, utilisé en tant
que FSBL. \\

\noindent
\bu{Timestamp : } voir Horodatage. \\

\noindent
\bu{TPIU : } (Trace Port Interface Unit) Composant puit, stockant les traces
Coresight dans une mémoire tampon. \\

\noindent
\bu{Trace : } Ensemble de paquets Coresight destinés à être analysés. \\

\noindent
\bu{UART : } (Universal Asynchronous Receiver Transmitter) Composant interface
entre un ordinateur et une liaison série. \\

\noindent
\bu{UBoot : } Logiciel Open Source utilisé comme chargeur d'amorçage ou SSBL.
\\
